
% ----- preamble -----

\documentclass[journal=mamobx,manuscript=suppinfo]{achemso}

\usepackage{lipsum}
\usepackage[dvipsnames]{xcolor}
\usepackage{graphicx}
\usepackage{amsmath, amssymb}
\usepackage{bm}
\usepackage{longtable}
\usepackage{multicol}
\DeclareMathOperator{\tr}{tr}
\DeclareMathOperator{\erf}{erf}
\DeclareMathOperator{\erfc}{erfc}
\raggedbottom
\usepackage{tikz}
\usetikzlibrary{positioning}
\usepackage{xr} % crossreferencing to main manuscript
\externaldocument[main:]{manuscript}

\newcommand{\ex}{\mathrm{ex}}
\newcommand{\txtf}{\mathrm{f}}
\newcommand{\txtb}{\mathrm{b}}
\newcommand{\twoD}{\mathrm{2D}}
\newcommand{\oneD}{\mathrm{1D}}
\newcommand{\wall}{\mathrm{wall}}
\newcommand{\bath}{\mathrm{bath}}

% Title and Author Info
\title{Mass-transfer driven spinodal decomposition in a ternary polymer solution}
\author{Douglas R. Tree}
\affiliation{Chemical Engineering Department, Brigham Young University, Provo, Utah}
\author{Lucas Francisco Dos Santos }
\affiliation{Universidade Estadual de Maring\'{a}, Maring\'{a}, Paran\'{a}, Brazil}
\author{Jan U. Garcia}
\affiliation{Chemical Engineering Department, University of California, Santa Barbara}
\author{Caden B. Wilson}
\affiliation{Chemical Engineering Department, Brigham Young University, Provo, Utah}
\author{Timothy R. Scott}
\affiliation{Chemical Engineering Department, Brigham Young University, Provo, Utah}
\author{Glenn H. Fredrickson}
\affiliation{Materials Research Laboratory, University of California, Santa Barbara}
\alsoaffiliation{Materials Department, University of California, Santa Barbara}
\alsoaffiliation{Chemical Engineering Department, University of California, Santa Barbara}
\date{\today}
\email{tree.doug@byu.edu}

% ----- document text -----
\begin{document}

\maketitle

\section{Hybrid Pseudo-Spectral/Finite-Difference Method}
% {{{

In a previous paper~\cite{Tree2017}, several of us outlined a method for solving the multi-fluid model represented by Equations~\ref{main:eq:diffusion}--\ref{main:eq:incompressibility} in the main text using pseudo-spectral methods and a semi-implicit time discretization scheme.
The combination of accuracy and stability provided by these methods allows one to efficiently solve the multi-fluid model equations in spite of the inherent disparity of length and time scales in the model.

The choice of a pseudo-spectral (PS) spatial discretization was key to the previous method's success.
As mentioned, PS methods give unparalleled accuracy, and Fourier transforms result in diagonal terms for linear differential operators, allowing a facile solution to the semi-implicit schemes that stabilize the time integration~\cite{Tree2017}.
Unfortunately, PS methods are limited to periodic boundary conditions (with a full Fourier transform) or homogeneous Dirichlet/Neumann conditions (with sine/cosine transforms), neither of which is sufficient to model a large nonsolvent bath~\cite{Hur2012}.

Alternatively, finite difference (FD) methods provide ample flexibility in treating boundary conditions, but are considerably less accurate than PS schemes.
Worse still, a multi-dimensional FD discretization results in a large, sparse matrix that is costly to invert when time-stepping with a semi-implicit method.

In an attempt to capture the major benefits of both approaches, we present here a hybrid PS/FD spatial discretization scheme where one dimension is treated via finite differences, and the remaining dimensions are discretized pseudo-spectrally.
The single FD dimension is able to provide time-dependent, inhomogeneous boundary conditions that are flexible enough to simulate the nonsolvent bath.
Additionally, this one-dimensional discretization gives a banded matrix that can be cheaply inverted in semi-implicit schemes.

Accordingly, numerical methods to solve both the convection-diffusion equation and the momentum equations (Eqs.~\ref{main:eq:diffusion}--\ref{main:eq:incompressibility}) using a hybrid PS/FD scheme to discretize space are described below.

\subsection{Diffusion Equation}
% {{{
It is convenient to examine the convection-diffusion equation, Eq.~\ref{main:eq:diffusion}, written explicitly in terms of the volume fractions, $\phi_{i}$.
To do so, we define diffusivity and gradient coefficient matrices,
\begin{gather}
D_{ij} = \sum_{k} M_{ik} H_{kj} \\
B_{ij} = \sum_{k} M_{ik} K_{kj} 
\end{gather}
where $M_{ij}$ is the mobility matrix, $H_{ij}$ is the Hessian matrix of second derivatives of the free energy functional and $K_{ij}$ is a diagonal matrix of the gradient coefficients~\cite{Tree2017}.
Substituting these definitions into Eq.~\ref{main:eq:diffusion} in the main text yields,
\begin{equation}
  \frac{\partial \phi_{i}}{\partial t} + \bm{v} \cdot \nabla \phi_{i} =
    \nabla \cdot \left [ \sum_{j}^{p,n} \left ( D_{ij} \nabla \phi_{j} 
    - B_{ij} \nabla \nabla^{2} \phi_{j} \right ) \right ] \label{eq:diffusion}
\end{equation}

To integrate Eq.~\ref{eq:diffusion} we need a method for the time integration in addition to a discretization of the spatial derivatives.
We approximate the time derivative via a semi-implicit method.
We begin by re-arranging Eq.~\ref{eq:diffusion} to a more useful form.
Positive and negative linear terms are introduced on the right-hand side giving,
\begin{equation} \label{eq:linear_terms}
\begin{split}
  \frac{\partial \phi_{i}}{\partial t} + \bm{v} \cdot \nabla \phi_{i} =
    & \nabla \cdot \left ( D_{ij} \nabla \phi_{j} \right )
    + D^{*}_{ij} \nabla^{2} \phi_{j} - D^{*}_{ij} \nabla^{2} \phi_{j} \\
    & - \nabla \cdot \left ( B_{ij} \nabla \nabla^{2} \phi_{j} \right )
    + B^{*}_{ij} \nabla^{4} \phi_{j} - B^{*}_{ij} \nabla^{4} \phi_{j} \textbf{.}
\end{split}
\end{equation}
Here, $D^{*}_{ij}$ and $B^{*}_{ij}$ are arbitrary constant matrices\footnotemark{}, and summation over repeated indices has been assumed in order to avoid cumbersome summation symbols.%
\footnotetext{In our previous work~\cite{Tree2017}, we found that a prudent choice of $D^{*}_{ij}$ and $B^{*}_{ij}$ led to an unconditionally stable time integration. Specifically, both matrices has to be (i) positive definite and (ii) have an eigenvalue at least as large as the largest eigenvalue in $D_{ij}(\phi_p, \phi_n)$ or $B_{ij}(\phi_p, \phi_{n})$ respectively.} %
Defining $\delta D_{ij} \equiv D_{ij} - D^{*}_{ij}$ and $\delta B_{ij} \equiv B_{ij} - B^{*}_{ij}$ as the difference between the concentration-dependent coefficient matrices and our chosen constant matrices, Eq.~\ref{eq:linear_terms} becomes,
\begin{equation}
  \frac{\partial \phi_{i}}{\partial t} + \bm{v} \cdot \nabla \phi_{i} =
    \nabla \cdot \left ( \delta D_{ij} \nabla \phi_{j} \right )
    + D^{*}_{ij} \nabla^{2} \phi_{j} 
    - \nabla \cdot \left ( \delta B_{ij} \nabla \nabla^{2} \phi_{j} \right )
    - B^{*}_{ij} \nabla^{4} \phi_{j}
\end{equation}
Finally, time is discretized with the non-linear terms being treated explicitly and the linear terms being treated implicitly.
Doing so gives,
\begin{equation} \label{eq:time_disc}
  \frac{\phi_{i}^{n+1} - \phi_{i}^{n}}{\Delta t} + \bm{v}^{n} \cdot \nabla \phi_{i}^{n} =
    \nabla \cdot \left ( \delta D_{ij}^{n} \nabla \phi_{j}^{n} \right )
    + D^{*}_{ij} \nabla^{2} \phi_{j}^{n+1} 
    - \nabla \cdot \left ( \delta B_{ij}^{n} \nabla \nabla^{2} \phi_{j}^{n} \right )
    - B^{*}_{ij} \nabla^{4} \phi_{j}^{n+1}
\end{equation}
which completes the semi-implicit time-stepping scheme.

To compute $\phi_{i}^{n+1}$ in Eq.~\ref{eq:time_disc}, the spatial derivatives must be discretized.
In our previous work, we approximated these derivatives using numerical Fourier transforms on a collocated grid, i.e.\ a pseudo-spectral approximation~\cite{Tree2017,Fredrickson2006}.
Presently, we will discretize the $x$-axis via finite differences and the $y$ and $z$ dimensions pseudo-spectrally.
Note that the choice of which dimension to treat via FD is arbitrary, so we can choose $x$ as the FD dimension without a loss of generality.

First, we discretize the implicit terms to isolate $\phi_{i}^{n+1}$. 
These terms contain second- and fourth-order gradient operators, which when separating the $x$ derivatives from the $y$ and $z$ derivatives are given by,
\begin{align}
  \nabla^{2} \phi_{j} &= \frac{\partial^2 \phi_{j}}{\partial x^2} + \nabla^{2}_{yz} \phi_{j} \\
  \nabla^{4} \phi_{j} & = \frac{\partial^{4} \phi_{j}}{\partial x^{4}} 
    + 2 \frac{\partial^{2}}{\partial x^{2}} \nabla^{2}_{yz} \phi_{j}
    + \nabla^{4}_{yz} \phi_{j}
\end{align}
where 
\begin{equation}
\nabla_{yz} = \left( \frac{\partial \phi_{j}}{\partial y}, \frac{\partial \phi_{j}}{\partial z} \right )^{\mathrm{T}}
\end{equation}
is a two-dimensional gradient for the PS directions only.
Applying a $yz$-Fourier transform to the operators gives,
\begin{align}
  \mathcal{F}_{yz}[\nabla^{2} \phi_{j}] &= \frac{\partial^2 \hat{\phi}_{j}}{\partial x^2} 
    - q^{2} \hat{\phi}_{j}  \label{eq:d2} \\
  \mathcal{F}_{yz}[\nabla^{4} \phi_{j}] & = \frac{\partial^{4} \hat{\phi}_{j}}{\partial x^{4}} 
    - 2 q^{2} \frac{\partial^{2} \hat{\phi}_{j}}{\partial x^{2}}
    + q^{4} \hat{\phi}_{j} \label{eq:d4} 
\end{align}
where the wavevector $\bm{q} = (q_{y}, q_{z})^{T}$, $q = |\bm{q}|$ and $\hat{\phi}_{j} = \mathcal{F}_{yz}[\phi_{j}]$.

With these expressions in hand, we now take the Fourier transform of Eq.~\ref{eq:time_disc}, and substitute Eq.~\ref{eq:d2} and Eq.~\ref{eq:d4} where appropriate.
This gives,
\begin{multline} \label{eq:FT}
  \frac{\hat{\phi}_{i}^{n+1} - \hat{\phi}_{i}^{n}}{\Delta t} + 
  \mathcal{F}_{yz}\left [ \bm{v}^{n} \cdot \nabla \phi_{i}^{n} \right ] = 
    \mathcal{F}_{yz}\left [ \nabla \cdot \left ( \delta D_{ij}^{n} \nabla \phi_{j}^{n} \right ) \right ] 
    - \mathcal{F}_{yz}\left [ \nabla \cdot \left ( \delta B_{ij}^{n} \nabla \nabla^{2} \phi_{j}^{n} \right ) \right ] \\
    + D^{*}_{ij} \left ( \frac{\partial^2 \hat{\phi}_{j}^{n+1}}{\partial x^2} - q^{2} \hat{\phi}_{j}^{n+1} \right )
    - B^{*}_{ij} \left ( \frac{ \partial^{4} \hat{\phi}_{j}^{n+1} }{\partial x^{4}}
    - 2 q^{2} \frac{\partial^{2} \hat{\phi}_{j}^{n+1}}{\partial x^{2}} + q^{4} \hat{\phi}_{j}^{n+1} \right )
\end{multline}
which upon solving for $\hat{\phi}_{j}^{n+1}$ yields,
\begin{multline} \label{eq:semiimplicit}
    \left [ I_{ij} 
    - \Delta t D^{*}_{ij} \left ( \frac{\partial^2 }{\partial x^2} - q^{2} \right )
    + \Delta t B^{*}_{ij} \left ( \frac{ \partial^{4} }{\partial x^{4}} 
                                  - 2 q^{2} \frac{\partial^{2}}{\partial x^{2}} 
                                  + q^{4} \right ) \right ] 
    \hat{\phi}_{j}^{n+1} = \\ 
      \hat{\phi}_{i}^{n}
    - \Delta t \mathcal{F}_{yz}\left [ \bm{v}^{n} \cdot \nabla \phi_{i}^{n} 
    + \nabla \cdot \left ( \delta D_{ij}^{n} \nabla \phi_{j}^{n} \right )
    - \nabla \cdot \left ( \delta B_{ij}^{n} \nabla \nabla^{2} \phi_{j}^{n} \right ) \right ] \mathrm{.}
\end{multline}

To evaluate the $x$-derivatives on the left-hand side of Eq.~\ref{eq:semiimplicit}, the $x$-direction is discretized into a regular grid with $M$ points and spacing $\Delta x$.
Finite difference formulas~\cite{Fornberg1988} are then substituted for the $x$-derivatives into Eq.~\ref{eq:semiimplicit}.
For example, centered, second-order finite difference formulas for an interior node at $x = x_{m}$ are given by,
\begin{gather}
  \frac{\partial^2 \hat{\phi}_{j,m}^{n+1}}{\partial x^2} \approx \frac{\hat{\phi}_{j, m+1}^{n+1} - 2 \hat{\phi}_{j, m}^{n+1} + \hat{\phi}_{j, m-1}^{n+1}}{\Delta x^{2}} \label{eq:d2_FD}\\
  \frac{\partial^4 \hat{\phi}_{j,m}^{n+1}}{\partial x^4} \approx \frac{\hat{\phi}_{j, m+2}^{n+1} - 4 \hat{\phi}_{j, m+1}^{n+1} + 6 \hat{\phi}_{j, m}^{n+1} - 4 \hat{\phi}_{j, m-1}^{n+1} + \hat{\phi}_{j, m-2}^{n+1}}{\Delta x^{4}}
\end{gather}
where the subscript $j$ indicates the component, the subscript $m$ indicates the node and the superscript $n+1$ indicates the time point.
More details about boundary conditions and the attendant complications of the treatment of the end nodes via finite differences can be found below.

Upon substitution of the FD formulas, Eq.~\ref{eq:semiimplicit} is of the form
\begin{equation} \label{eq:big_linear}
\bm{A}_{ij} \hat{\phi}_{j}^{n+1} = \bm{b}_{i}
\end{equation}
where
\begin{equation}
\bm{A}_{ij} = \left [ I_{ij} 
    - \Delta t D^{*}_{ij} \left ( \frac{\partial^2 }{\partial x^2} - q^{2} \right )
    + \Delta t B^{*}_{ij} \left ( \frac{ \partial^{4} }{\partial x^{4}} 
                                  - 2 q^{2} \frac{\partial^{2}}{\partial x^{2}} 
                                  + q^{4} \right ) \right ] 
\end{equation}
is a square $\mathcal{M}\times\mathcal{M}$ block-banded matrix where $\mathcal{M} = N M$ with $N$ being the number of components and $M$ (as defined above) is the number of grid points spanning the $x$ direction.
The right hand side is the $\mathcal{M} \times 1$ nonlinear vector function,
\begin{equation} \label{eq:rhs}
\bm{b}_{i} = f(\hat{\phi}_{i}^{n}) = 
      \hat{\phi}_{i}^{n}
    - \Delta t \mathcal{F}_{yz}\left [ \bm{v}^{n} \cdot \nabla \phi_{i}^{n} 
    + \nabla \cdot \left ( \delta D_{ij}^{n} \nabla \phi_{j}^{n} \right )
    - \nabla \cdot \left ( \delta B_{ij}^{n} \nabla \nabla^{2} \phi_{j}^{n} \right ) \right ] \mathrm{.}
\end{equation}

The right-hand side vector, Eq.~\ref{eq:rhs}, can be evaluated at each FD node using a combination of forward and inverse fast-Fourier transforms and finite difference formulas using the known quanitity $\phi_{j}^{n}$.
For example, the second-order gradient can be decomposed into FD and PS derivatives,
\begin{equation} \label{eq:2nd_order}
\nabla \cdot ( \delta D_{ij}^{n} \nabla \phi_{j}^{n} )
  =  \frac{\partial}{\partial x} \left ( \delta D_{ij}^{n} \frac{\partial \phi_{j}^{n}}{\partial x} \right )
     + \nabla_{yz} \cdot ( \delta D_{ij}^{n} \nabla_{yz} \phi_{j}^{n}) \mathrm{.}
\end{equation}
Because of the non-constant coefficients ${\delta D}_{ij}^{n}$ and ${\delta B}_{ij}^{n}$, care must be taken to use self-adjoint finite difference operators for the x-derivative on the right-hand side of Eq.~\ref{eq:2nd_order}~\cite{FordVersypt2014}.
In this case, the recursive, second-order, self-adjoint, centered finite difference formula
\begin{equation}
  \frac{d f_{m}}{d x} = \frac{f_{m+1/2}-f_{m-1/2}}{\Delta x} + \textrm{O}(\Delta x^{2})
\end{equation}
results in
\begin{equation}
  \frac{\partial}{\partial x} \left ( \delta D_{ij, \, m}^{n} \frac{\partial \phi_{j}^{n}}{\partial x} \right )
     =  \frac{\delta D_{ij, \,m+\frac{1}{2}}^{n} (\phi_{j, m+1}^{n} - \phi_{j,m}^{n}) - \delta D_{ij, m-\frac{1}{2}}^{n} (\phi_{j, m}^{n} - \phi_{j,m-1}^{n})}{\Delta x^{2}} \mathrm{.}
\end{equation}
for each node $m$, where $\delta D_{ij, \,m+\frac{1}{2}}^{n}$ and $\delta D_{ij, m-\frac{1}{2}}^{n}$ can be obtained via linear interpolation.
The second term on the right-hand side of Eq.~\ref{eq:2nd_order} can be evaluated using repeated forward and inverse Fourier transforms, for example,
\begin{equation}
  \mathcal{F}_{yz} [ \nabla_{yz} \cdot ( \delta D_{ij}^{n} \nabla_{yz} \phi_{j}^{n}) ] = 
    i \bm{q} \cdot \mathcal{F}_{yz}\left [ \delta D_{ij}^{n} \mathcal{F}_{yz}^{-1}[i \bm{q} \hat{\phi}_{j}^{n}]  \right ] \mathrm{.}
\end{equation}
The method for evaluating the fourth-order gradient term and the edge nodes is similar to the example shown here, noting that self-adjoint finite difference formulas are needed in both cases.

Finally, once $\bm{A}_{ij}$ and $\bm{b}_{i}$ have been calculated (including the boundary conditions), the linear equation (Eq.~\ref{eq:big_linear}) can be solved.
Because $\bm{A}_{ij}$ is banded, it can be efficiently solved in O($\mathcal{M}$) time, which was one of the primary motivating factors for using a hybrid FD/PS method from the outset.
% }}}

\subsection{Momentum Equation}
% {{{
The main difficulty in solving the momentum equation (Eq.~\ref{main:eq:momentum} in the main text) is (i) satisfying Eq.~\ref{main:eq:incompressibility}, the incompressibility constraint, and (ii) efficiently dealing with a concentration-dependent viscosity.
In our previous PS-only method~\cite{Tree2017}, (i) was accomplished by eliminating the incompressibility constraint via the transverse projection operator, which can be explicitly written in terms of the wavevector $\bm{q}$ in Fourier space.
Difficulty (ii) was overcome by using an iterative Picard's method modified with a first-order continuation guess and Anderson mixing to accelerate convergence.

Moving to a hybrid FD/PS method, the transverse projection operator is no longer a viable option for satisfying incompressibility.
Instead, we derive a so-called ``pressure--Poisson'' equation~\cite{Rempfer2006} that allows explicit access to the pressure and a subsequent solution to Eq.~\ref{main:eq:momentum}.
Fortunately, Picard's method (accompanied by continuation and Anderson mixing) is still an effective method for dealing with a concentration-dependent viscosity.

Anticipating the need to formally solve for $\bm{v}$ for a future Picard iteration, the first step in the derivation of the numerical method involves the addition and subtraction of a linear viscous dissipation term to Eq.~\ref{main:eq:momentum},
\begin{equation} \label{eq:mom_first}
  0 = -\nabla p - \nabla \cdot \bm{\Pi} + 
    \nabla \cdot \left [ \eta (\nabla \bm{v} + \nabla \bm{v}^{T}) \right ] 
    + \eta^{*} \nabla^{2} \bm{v} - \eta^{*} \nabla^{2} \bm{v}
\end{equation}
where $\eta^{*}$ is an arbitrary constant\footnotemark{} chosen for stability in the Picard iteration. %
\footnotetext{Our numerical testing indicates that choosing $\eta^{*} = \max(\eta)$ results in an unconditionally stable iteration.}
For convenience, we define an excess stress tensor as the viscous stress due to the local difference in viscosity from $\eta^{*}$,
\begin{equation} \label{eq:excess_stress}
  \bm{\sigma}_{\ex} \equiv (\eta - \eta^{*} ) (\nabla \bm{v} + \nabla \bm{v}^{T})
\end{equation}
which upon substitution into Eq.~\ref{eq:mom_first} and re-arrangement gives,
\begin{equation} \label{eq:picard_momentum}
  \eta^{*} \nabla^{2} \bm{v} = \nabla \cdot \left ( p\bm{I} + \bm{\Pi} - \bm{\sigma}_{\ex} \right )
\end{equation}
By taking the divergence of Eq.~\ref{eq:picard_momentum}, one obtains a pressure-Poisson equation
\begin{gather} \label{eq:pressure_poisson}
  \nabla^{2} p = \nabla \nabla : \left ( \bm{\sigma}_{\ex} - \bm{\Pi} \right )
\end{gather}
noting that the term on the left-hand side is eliminated using incompressibility, $\nabla \cdot \bm{v} = 0$.

Equations~\ref{eq:excess_stress}--\ref{eq:pressure_poisson} represent a complete statement of momentum conservation and incompressibility, and can be used to solve for the velocity at a given time\footnotemark{}.%
\footnotetext{Note that unlike the diffusion equation, no time derivative appears in Equations~\ref{eq:excess_stress}--\ref{eq:pressure_poisson}. Instead, the time-dependence is implicit through the appearance of the volume-fraction in the osmotic stress and viscosity.} %
This is done by discretizing the gradient operators and formally solving for $p$ and $\bm{v}$ (remember that $\bm{\sigma}_{\ex}$ depends on $\bm{v}$).
Both Eqs.~\ref{eq:picard_momentum} and~\ref{eq:pressure_poisson} contain the Laplacian operator, which is given in Eq.~\ref{eq:d2} above.
Expanding the Laplacian operators on the left-hand side and taking a $yz$-Fourier transform gives,
\begin{gather}
  \left ( \frac{\partial^2}{\partial x^2} - q^{2} \right ) \hat{p} =  \mathcal{F}_{yz} \left [ \nabla \nabla : \left ( \bm{\sigma}_{\ex} - \bm{\Pi} \right ) \right ] \\
  \left ( \frac{\partial^2}{\partial x^2} - q^{2} \right ) \hat{\bm{v}} =  \frac{1}{\eta^{*}} \mathcal{F}_{yz} \left [ \nabla \cdot \left ( p\bm{I} + \bm{\Pi} - \bm{\sigma}_{\ex} \right ) \right ]
\end{gather}
Using the finite difference formula in Eq.~\ref{eq:d2_FD} for the $x$-derivatives, these discretized equations reduce to a system of nonlinear equations of the form,
\begin{gather} 
\bm{B} \hat{p} = g(\hat{\phi}_{i}^{n+1}, \hat{\bm{v}}) \\
\bm{C} \hat{\bm{v}} = \bm{h}(\hat{\phi}_{i}^{n+1}, \hat{\bm{v}}, \hat{p})
\end{gather}
where $\bm{B}$ is an $M \times M$ banded matrix and \bm{$C$} is a $3M \times 3M$ banded matrix.
Because both matrices are banded, they can be solved in O($M$) time.

The coupled set of equations are iteratively solved using Picard's method,
\begin{gather} 
\bm{B} \hat{p}^{k+1} = g(\hat{\phi}_{i}^{n+1}, \hat{\bm{v}}^{k}) \label{eq:pressure_poisson_matrix} \\
\bm{C} \hat{\bm{v}}^{k+1} = \bm{h}(\hat{\phi}_{i}^{n+1}, \hat{\bm{v}}^{k}, \hat{p}^{k}) \label{eq:picard_momentum_matrix}
\end{gather}
where $k$ denotes the iterative index.
The method proceeds as follows:
\begin{enumerate}
  \item The divergence of the osmotic stress tensor, $\nabla \cdot \bm{\Pi}$ (Eq.~\ref{main:eq:div_osmotic_stress} in the main text), is calculated using the volume fractions obtained from the diffusion equation, $\phi_{i}^{n+1}$.
  \item A velocity guess, $\bm{v}^{k}$ is obtained from a continuation method based on the velocities at previous timesteps, $\bm{v}^{n}$, $\bm{v}^{n-1}$, \ldots, or previous iterations $\bm{v}^{k}$, $\bm{v}^{k-1}$, \ldots, and is used to calculate the divergence of the excess viscous stress, $\nabla \cdot \bm{\sigma}_{\ex}$.
  \item $\nabla \cdot \bm{\Pi}$ and $\nabla \cdot \bm{\sigma}_{\ex}$ are used to evaluate the right-hand side of Eq.~\ref{eq:pressure_poisson_matrix}, and the (now-linear) system is solved for the pressure $p^{k+1}$.
  \item The right-hand side of Eq.~\ref{eq:picard_momentum_matrix} is evaluated, and the linear system is solved for a new velocity, $\bm{v}^{k+1}$.
  \item The norm of the difference between velocity vectors, $| \bm{v}^{k+1} - \bm{v}^{k} |$ is compared to a specified tolerance (in our case, typically $10^{-6}$).
If the norm is small enough, the iteration terminates.
If not, one returns to step two\footnotemark{} and performs another iteration. %
\footnotetext{The osmotic stress tensor is not a function of $\bm{v}$ or $p$ and therefore only needs to be calculated once.}
\end{enumerate}

Finally, for completeness, we provide some details regarding the evaluation of the functions,
\begin{gather}
g(\hat{\phi}_{i}^{n+1}, \hat{\bm{v}}^{k}) = \mathcal{F}_{yz} \left [ \nabla \nabla : \left ( \bm{\sigma}_{\ex} - \bm{\Pi} \right ) \right ] \label{eq:poisson_rhs} \\
\bm{h}(\hat{\phi}_{i}^{n+1}, \hat{\bm{v}}^{k}, \hat{p}^{k}) = \frac{1}{\eta^{*}} \mathcal{F}_{yz} \left [ \nabla \cdot \left ( p\bm{I} + \bm{\Pi} - \bm{\sigma}_{\ex} \right ) \right ] \label{eq:momentum_rhs}
\end{gather}
from the right-hand sides of Eqs.~\ref{eq:pressure_poisson_matrix} and \ref{eq:picard_momentum_matrix}.
For the most part, this process is straightforward, and these functions are evaluated using principles similar to those discussed for the diffusion equation. 
Namely, one must use (i) self-adjoint finite difference operators for the $x$-derivatives and (ii) repeated forward and reverse $yz$-Fourier transforms for the $y$ and $z$ derivatives.

However, the expansion of the derivative operators in Eqs.~\ref{eq:poisson_rhs} and \ref{eq:momentum_rhs} is more complex than those on the right-hand side of the diffusion equation, so we document them here.
Expanding the terms in $g$ gives,
\begin{align}
  \nabla \nabla : \bm{\sigma}_{\ex} 
    & = \nabla\nabla : (\delta \eta \nabla \bm{v}) + 
      \nabla\nabla : (\delta \eta (\nabla \bm{v})^{T}) \\
    \begin{split}
    & = 2 \frac{\partial^{2}}{\partial x^{2}} \left ( \delta \eta \frac{\partial v_{x}}{\partial x} \right ) 
      + 2 \nabla_{yz} \cdot \frac{\partial}{\partial x} \left ( \delta \eta \frac{\partial \bm{v}_{yz}}{\partial x} \right ) \\
    & \quad + 2 \nabla_{yz} \cdot \frac{\partial}{\partial x} \left ( \delta \eta \nabla_{yz} {v}_{x} \right )
      + \nabla_{yz} \nabla_{yz} : \left ( \delta \eta \nabla_{yz} \bm{v}_{yz} \right )  \textrm{, } \\ 
    \end{split}
\end{align}
and
\begin{align}
  \nabla \nabla : \bm{\Pi}
    & = \nabla \cdot \left ( \phi_{i} H_{ij} \nabla \phi_{j} \right )
      - \nabla \cdot \left ( \phi_{i} K_{ij} \nabla \nabla^{2} \phi_{j} \right ) \\
    \begin{split}
    & = \frac{\partial}{\partial x} \left ( \phi_{i} H_{ij} \frac{\partial \phi_{j}}{\partial x} \right ) 
       + \nabla_{yz} \cdot \left ( \phi_{i} H_{ij} \nabla_{yz} \phi_{j} \right ) \\
    &\quad  - \frac{\partial}{\partial x} \left ( \phi_{i} K_{ij} \frac{\partial^{3} \phi_{j} }{\partial x^{3}} \right ) 
            - \frac{\partial}{\partial x} \left ( \phi_{i} K_{ij} \frac{\partial}{\partial x} \nabla_{yz}^{2} \phi_{j} \right ) \\
    &\quad  - \nabla_{yz} \cdot \left ( \phi_{i} K_{ij} \nabla_{yz} \frac{\partial^{2} \phi_{j} }{\partial x^{2}} \right )
            - \nabla_{yz} \cdot \left ( \phi_{i} K_{ij} \nabla_{yz} \nabla_{yz}^{2} \phi_{j} \right ) \textrm{.}
      \end{split}
\end{align}
where $\delta \eta = \eta - \eta^{*}$ and $\bm{v}_{yz} = \left ( v_{y}, v_{z} \right )^{T}$.
Expanding the terms in $\bm{h}$ gives,
\begin{equation}
  \nabla p = \left ( \frac{\partial p}{\partial x}, \nabla_{yz} p \right )^{T}
\end{equation}
\begin{align}
  \nabla \cdot \bm{\Pi} & = \phi_{i} H_{ij} \nabla \phi_{j} - \phi_{i} K_{ij} \nabla \nabla^{2} \phi_{j} \\
  \begin{split}
    & = \left ( \phi_{i} H_{ij} \frac{\partial \phi_{j} }{\partial x} 
                - \phi_{i} K_{ij} \frac{\partial^{3} \phi_{j} }{\partial x^{3}}
                - \phi_{i} K_{ij} \nabla_{yz}^{2} \frac{\partial \phi_{j} }{\partial x},  \right . \\
    & \qquad \left . \phi_{i} H_{ij} \nabla_{yz} \phi_{j} 
                - \phi_{i} K_{ij} \nabla_{yz} \frac{\partial^{2} \phi_{j}}{\partial x^{2}}
                - \phi_{i} K_{ij} \nabla_{yz} \nabla_{yz}^{2} \phi_{j} \right )^{T} \\
  \end{split}
\end{align}
and
\begin{align}
  \nabla \cdot \bm{\sigma}_{\ex} & = \nabla \cdot [\delta \eta (\nabla \bm{v} + \nabla \bm{v}^{T}) ] \\
    \begin{split}
    & = \left ( 2 \frac{\partial}{\partial x} ( \delta \eta \frac{\partial v_{x}}{\partial x} ) \right .
      + \nabla_{yz} \cdot ( \delta \eta \frac{\partial \bm{v}_{yz}}{\partial x} )
      + \nabla_{yz} \cdot ( \delta \eta \nabla_{yz} {v}_{x} ), \\
    & \qquad \left . \frac{\partial}{\partial x} ( \delta \eta \frac{\partial \bm{v}_{yz}}{\partial x} )
      + \frac{\partial}{\partial x} \left ( \delta \eta \nabla_{yz} {v}_{x} \right )
      + \nabla_{yz} \cdot [ \delta \eta (\nabla_{yz} \bm{v}_{yz} + \nabla_{yz} \bm{v}_{yz}^{T} ) ] \right )^{T} \textrm{, } \\ 
    \end{split}
\end{align}

Numerically evaluating these expressions involves the use of both traditional and self-adjoint finite differences to approximate the $x$-derivatives, and repeated forward and inverse $yz$-Fourier transforms where 
\begin{gather}
  \mathcal{F}_{yz}[\nabla_{yz}] = \left ( i q_y, i q_z \right )^{T} \\
  \mathcal{F}_{yz}[\nabla_{yz}\nabla_{yz}] = \begin{pmatrix}
    - q_y^2 & -q_y q_z \\
    -q_y q_z & -q_z^{2}
  \end{pmatrix}
\end{gather}
are the pseudo-spectral $yz$-gradient operators.
Finally, note that it is often convenient to exchange the order of operations between the Fourier transforms, $x$-derivatives and $yz$-derivatives, which is permitted since all are linear operators.
% }}}

\subsection{Boundary Conditions} \label{sec:BCs}
% {{{
As desired, the hybrid FD/PS method allows the implementation of boundary conditions for both the diffusion and momentum equations.
For the thin-film geometry, the diffusion equation has two boundary conditions at each boundary (wall and bath).
At the wall, we have a mass-conserving, no-flux condition
\begin{equation}
M_{ij} \left . \frac{d \mu_{i}}{d x} \right |_{x = \mathrm{wall}} = 0 \textrm{,}
\end{equation}
and a prescription of the local contact angle 
\begin{equation}
\left . \frac{d \phi_{i}}{d x} \right |_{x = \mathrm{wall}} = -\frac{\chi_{wi}}{\kappa}
\end{equation}
where $\chi_{wi}$ is the interaction coefficient between component $i$ and the wall.
The conditions in the bath are given by
\begin{gather}
\lim_{x\rightarrow\infty} \phi_{i} = \phi_{i}^{\txtb} \\
\lim_{x\rightarrow\infty} \frac{d \phi_{i}}{dx} = 0 \textrm{.}
\end{gather}

For the momentum equation, we have boundary conditions for the velocity only~\cite{Rempfer2006}.
At the wall, we have the customary no-slip and no-penetration conditions, $\bm{v} = 0$.
At the edge of the bath, we also require $\bm{v} = 0$, meaning the bath must still be large enough for all of the relevant hydrodynamics to decay to zero before reaching the bath ``boundary.''
Future work with these methods could go further to implement ``open'' boundary conditions obviating the need for such a requirement.

In theory it is possible for a simulation of the thin film geometry with the above boundary conditions to have a $2\times$ speedup over an identical simulation using periodic conditions, due to a reduction in the needed domain size.
However, in practice we find that finite difference derivatives require greater resolution than pseudospectral derivatives, eliminating the potential benefit.
We also found that our PS code design was inherently faster, due to the relative easy and relatively low complexity compared to the hybrid FD/PS code.
As such, all of the simulations of the thin film geometry in the present work were performed with the PS method.

However, long-time simulations require time-dependent boundary conditions, which can only be achieved via the hybrid FD/PS code.
In these calculations, a 1D periodic simulation is first performed with a very large bath (typically $Lx = 4092\, R_{0}$).
Two ``boundaries'' are then chosen: the axis of symmetry (the wall) and a point in the bath near to the eventual interface between the film and bath.
The latter ($x_{\bath}$) is chosen by convenience to be far away from the phase separating interface, but small enough to lead to a significant reduction in run-time.
To verify that the choice of $x_{\bath}$ did not result in significant errors, concentrations from large-bath 1D simulations were compared to concentrations of 1D projections from the higher dimensional simulations.

At these boundaries, we proscribe,
\begin{gather}
\phi_{i}^{\twoD}(x_{\wall}, t) = \phi_{i}^{\oneD}(x_{wall}, t) \\
\left . \frac{d \phi_{i}^{\twoD}(x,t)}{dx}\right |_{x=x_{\wall}} = \left . \frac{d \phi_{i}^{\oneD}(x,t)}{dx}\right |_{x=x_{wall}} \\
\phi_{i}^{\twoD}(x_{\bath}, t) = \phi_{i}^{\oneD}(x_{\bath}, t) \\
\left . \frac{d \phi_{i}^{\twoD}(x,t)}{dx}\right |_{x=x_{\bath}} = \left . \frac{d \phi_{i}^{\oneD}(x,t)}{dx}\right |_{x=x_{\bath}} 
\end{gather}
for the volume fractions and
\begin{gather}
\bm{v}(x_{\wall}, t) = 0 \\
\bm{v}(x_{\bath}, t) = 0
\end{gather}
for the velocities.
Again, when resolving hydrodynamics we require that the boundary be sufficiently far from the interface, so the velocity decays to zero to avoid an unphysical condition.

At the root of the hybrid FD/PS method, both the diffusion equation, Eq.~\ref{eq:big_linear}, and the momentum equation, Eq.~\ref{eq:picard_momentum_matrix} are block-diagonal, linear matrix equations.
Boundary conditions can be added by altering the elements of the matrix which correspond to the boundary grid point.
We use straightforward finite difference techniques to discretize these boundary conditions.

For periodic boundary conditions, we use centered finite differences for all derivatives,
\begin{gather}
\left . \frac{df}{dx} \right |_{x_{m}} = \frac{f_{m+1} - f_{m-1}}{2 \Delta x} \\
\left . \frac{d^{2}f}{dx^{2}} \right |_{x_{m}} = \frac{f_{m+1} - 2 f_{m} + f_{m-1}}{(\Delta x)^{2}} \\
\left . \frac{d^{3}f}{dx^{3}} \right |_{x_{m}} = \frac{f_{m+2} - 2 f_{m+1} + 2 f_{m-1} - f_{m-2}}{2 (\Delta x)^{3}} \\
\left . \frac{d^{4}f}{dx^{4}} \right |_{x_{m}} = \frac{f_{m+2} - 4 f_{m+1} + 6 f_{m} - 4 f_{m-1} + f_{m-2}}{(\Delta x)^{4}}
\end{gather}
where $x_{m}$ is the location of the gridpoint in the x-direction.
As is standard, calls to off-grid locations are ``wrapped around'' the boundary.
For example if $i = 0$ on a grid with an upper limit of $N_{x}-1$ points, then the finite difference formula for the third derivative would be,
\begin{equation}
\left . \frac{d^{3}f}{dx^{3}} \right |_{x_{0}} = \frac{f_{2} - 2 f_{1} + 2 f_{Nx-2} - f_{Nx-3}}{2 (\Delta x)^{3}}\textrm{.}
\end{equation}

For non-periodic boundary conditions, we use one-sided finite differences~\cite{Fornberg1988}.
On the left-side boundary we use,
\begin{gather}
\left . \frac{df}{dx} \right |_{x_{m}} = \frac{-\frac{1}{2} f_{m+2} + 2 f_{m+1} - \frac{3}{2} f_{m}}{\Delta x} \\
\left . \frac{d^{2}f}{dx^{2}} \right |_{x_{m}} = \frac{-f _{m+3} + 4 f_{m+2} - 5 f_{m+1} + 2 f_{m}}{(\Delta x)^{2}} \\
\left . \frac{d^{3}f}{dx^{3}} \right |_{x_{m}} = \frac{-\frac{3}{2} f_{m+4} + 7 f_{m+3} - 12 f_{m+2} + 9 f_{m+1} - \frac{5}{2} f_{m}}{(\Delta x)^{3}} \\
\left . \frac{d^{4}f}{dx^{4}} \right |_{x_{m}} = \frac{-2 f_{m+5}  + 11 f_{m+4} - 24 f_{m+3} + 26 f_{m+2} - 14 f_{m+1} + 3 f_{m}}{(\Delta x)^{4}}
\end{gather}
and on the right-side we use,
\begin{gather}
\left . \frac{df}{dx} \right |_{x_{m}} = \frac{\frac{1}{2} f_{m-2} - 2 f_{m-1} + \frac{3}{2} f_{m}}{\Delta x} \\
\left . \frac{d^{2}f}{dx^{2}} \right |_{x_{m}} = \frac{-f _{m-3} + 4 f_{m-2} - 5 f_{m-1} + 2 f_{m}}{(\Delta x)^{2}} \\
\left . \frac{d^{3}f}{dx^{3}} \right |_{x_{m}} = \frac{\frac{3}{2} f_{m-4} - 7 f_{m-3} + 12 f_{m-2} + 9 f_{m-1} + \frac{5}{2} f_{m}}{(\Delta x)^{3}} \\
\left . \frac{d^{4}f}{dx^{4}} \right |_{x_{m}} = \frac{-2 f_{m-5}  + 11 f_{m-4} - 24 f_{m-3} + 26 f_{m-2} - 14 f_{m-1} + 3 f_{m}}{(\Delta x)^{4}}
\textrm{.}
\end{gather}

While the principles are simple, implementing boundary conditions requires tedious bookkeeping.
Accordingly, we illustrate one example where we apply the boundary conditions
\begin{gather}
\phi_{p}(x=0) = \phi_{p}^{\txtb} \qquad \left . \frac{d \phi_{p}}{dx} \right |_{x=0} = 0  \\
\phi_{n}(x=0) = \phi_{n}^{\txtb} \qquad \left . \frac{d \phi_{n}}{dx} \right |_{x=0} = 0 
\end{gather}
to the diffusion equation, Eq.~\ref{eq:big_linear}.
We do not show an example using the momentum equation, but the procedure is identical.
The first block row of $\bm{A}_{ij} \phi_{j} = \bm{b}_{i}$ contains the first boundary condition and is given by
\begin{equation}
\begin{bmatrix}
  1 & 0 & \ldots \\
  0 & 1 & \ldots \\
\end{bmatrix}
\begin{bmatrix}
  \phi_{p, 0} \\
  \phi_{n, 0} \\
  \vdots
\end{bmatrix}
=
\begin{bmatrix}
  \phi_{p}^{\txtb} \\
  \phi_{n}^{\txtb}
\end{bmatrix}
\textrm{.}
\end{equation}
The second boundary condition is given on the second block row,
\begin{equation}
\begin{bmatrix}
  -\frac{3}{2} & 0 & 2 & 0 & -\frac{1}{2} & 0 & \ldots \\
  0 & -\frac{3}{2} & 0 & 2 & 0 & -\frac{1}{2} & \ldots \\
\end{bmatrix}
\begin{bmatrix}
  \phi_{p, 0} \\
  \phi_{n, 0} \\
  \phi_{p, 1} \\
  \phi_{n, 1} \\
  \phi_{p, 2} \\
  \phi_{n, 2} \\
  \vdots
\end{bmatrix}
=
\begin{bmatrix}
  0 \\
  0
\end{bmatrix}
\textrm{.}
\end{equation}
Note that applying these boundary conditions requires altering both the right and left hand side of Eq.~\ref{eq:big_linear}.

Non-periodic cases of Eq.~\ref{eq:big_linear} and Eq.~\ref{eq:picard_momentum_matrix} can be solved using a block-banded matrix solver.
However, when using periodic boundary conditions with the hybrid FD/PS method, both matrices become cyclic.
In this situation, we use the Sherman-Morrison-Woodbury formula~\cite{Press2002}, which allows one to get rid of the off-diagonal block elements and recover an O$(\mathcal{M})$ method.
% }}}

% }}}

\section{Theory of Surface-Directed Spinodal Decomposition}
% {{{
Despite the fact that we do not have a direct solution of surface-directed spinodal decomposition for our ternary Flory-Huggins model, we can still be quantitative and use a prediction of the front velocity $v^{*}$ and dominant wavenumber $q^{*}$ that Ball and Essery obtain based solely on the dispersion relation of the linearized theory.
Using the so-called marginal stability hypothesis~\cite{Ball1990, vanSaarloos1987, vanSaarloos1988}, they predict that the front velocity and dominant wave number are,
\begin{gather}
v^{*} \approx 4.588 M \kappa q_{m}^{3} \label{eq:front_velocity} \\
q^{*} \approx 1.083 q_{m} \label{eq:wave_number}
\end{gather}
where $q_{m}$ is the fastest growing mode from the linear stability analysis.

In a previous publication~\cite{Tree2017}, we looked at perturbations about a homogeneous state inside the spinodal using the model described in Eq.~\ref{main:eq:1D_diffusion} in the main text.
Assuming a constant scalar mobility $M$ and pseudo-binary parameters, the dynamics of the most unstable eigenmode of the system gives an identical dispersion relation to that obtained by Ball and Essery, which in our notation is expressed as the largest eigenvalue,
\begin{equation}
\lambda_{+}(q) = M \kappa q^{2} (2 q_{m}^{2} - q^{2} ) \textrm{.}
\end{equation}
Because the dispersion relations are identical, Equations~\ref{eq:front_velocity} and ~\ref{eq:wave_number} remain valid for our system, where the fastest growing mode is given by
\begin{equation}
q_{m}^{2} = 
  \frac{1}{4 \kappa} \left [ -\left ( \frac{1}{N \phi_{p}} + \frac{2}{\phi_{s}} + \frac{1}{\phi_{n}} \right )
  + \sqrt{ \left ( \frac{1}{N \phi_{p}} - \frac{1}{\phi_{n}} \right )^{2} 
    + 4 \left ( \frac{1}{\phi_{s}} + \chi \right ) 
  } \right ] \textrm{.}
\end{equation}

Compared to the previous theoretical results, the above analysis of surface directed spinodal decomposition assumes the initial condition is prepared in an unstable state.
Diffusion from the bulk is present via a boundary condition, but it does not drive the phase separation.
Additionally, the analysis is limited to early-time behavior; we cannot infer anything about coarsening processes that happen at late times.

Surface-directed spinodal decomposition also introduces new timescales.
The timescale for propagation of the spinodal wave through the film is given by $l_{\txtf}/v^{*} \sim l_{\txtf}/(M \kappa q_{m}^{3})$.
However, if noise is present (even if only in the initial condition), the bulk of the film is expected to undergo isotropic spinodal decomposition away from the interface at a time $\lambda_{+}(q_{m})^{-1} = 1/(M \kappa q_{m}^{4})$.
Combining this rate and the wave velocity gives a prediction of a length scale, $1/q_{m}$, at which a crossover should happen between surface-directed and bulk spinodal decomposition.
While this length scale is the same as the dominant wavelength, the numerical prefactor will surely differ since the crossover length is sensitive to the strength of the density fluctuations and properties of the interface~\cite{Ball1990}.

% }}}

\section{Data Tables} 
% {{{
% Tables of data that were run?
% State how the 2D data needed noise to be added to observe droplets?

As discussed in the methods section, the data contained in the paper is from multiple simulation data sets.
For the early-time regime 1D, 2D and 3D simulations were performed with periodic boundary conditions for $t = \{0, 2, \ldots, 100\}\tau_{R}$ where $\tau_{R}$ is the Rouse time of the reference polymer~\cite{Tree2017}.
In 1D simulations the domain was large ($L_{x} = 4096 R_{0}$) and the bath/film interface was set in the middle of the domain ($f = 0.5$).
Recall that an initial condition which is symmetric about $x$ must be used to satisfy the periodic boundary conditions, as in Figure~\ref{main:fig-schematic}, halving the usable domain size.
By necessity, 2D and 3D simulations were run in smaller domains: $L_{x} = 512 R_{0}$, $L_{y} = 256 R_{0}$ for 2D and $L_{x} = 64 R_{0}$, $L_{y} = 64 R_{0}$, $L_{z} = 64 R_{0}$ for 3D.
In all cases, the plane wave resolution was $2\times R_{0}$, i.e.\ $N_{x} = 2 L_{x}$, etc.
% Note to self: The actual periodic simulation Caden ran swapped Lx for Ly, but I used what I did here to be consistent with the rest of the paper. 

For the late-time regime 1D simulations were also performed with periodic boundary conditions in a very long domain ($L_{x} = 4096 R_{0}$) for times $t  = \{0, 100, \ldots, 5000\} \tau_{R}$.
Unlike the short-time calculations, the initial film thickness was varied for the longer time runs with $f = \{0.025, 0.05, 0.1\}$ where $l_{f} = f L_{x}/2$.
2D and 3D simulations were performed with a truncated bath where an accompanying 1D simulation served as both the initial condition and the time-dependent bath boundary condition.
Again, 2D and 3D simulations were run in smaller domains than were possible in 1D: $L_{x} = 128 R_{0}$, $L_{y} = 128 R_{0}$ for 2D and $L_{x} = 128 R_{0}$, $L_{y} = 64 R_{0}$, $L_{z} = 32 R_{0}$ for 3D.
For the 2D and 3D simulations with proscribed boundary conditions, the resolution was increased to $4\times R_{0}$ to account for the lower accuracy of finite difference formulas relative to pseudospectral derivatives.

Below, we give three tables which summarize specific parameters where the data was taken.
Table~\ref{tab:energy_params} gives the free energy model (ternary Flory-Huggins) parameters which were used.
We have tried to be explicit in the main text, but unless otherwise noted, data were taken using the ``Base Case'' given in the first line of the table.
The remaining two tables show the initial film concentrations used in various simulations;
Table~\ref{tab:coarse} shows a coarse sweep of composition space and Table~\ref{tab:fine} shows a finer sweep.
All of the initial film compositions used in the main text come from one of these two tables (e.g. Figure~\ref{main:fig-inf_domain}, Figure~\ref{main:fig-early_time_morphology}, and Figure~\ref{main:fig-late_time_regimes}).
To aid the reader, Figure~\ref{fig-ic_fractions} shows both of these sets of initial conditions plotted in composition space with their corresponding run number.

\begin{table}
\caption{Energy model parameters used in all simulations. 
In all cases $\kappa_{pn} = \kappa_{ps}$, $\chi_{ps} = 0$ and $N_{s} = N_{n} = 1$.
}
\label{tab:energy_params}
\bgroup
\def\arraystretch{1.3}%
\begin{tabular}{ccccc}
\hline
ID & $N_{p}$ & $\chi_{pn}$ & $\chi_{ns}$ & $\kappa$ \\
\hline
``Base case'' & 20 & 1.04805 & 0 & 2 \\
``N=10'' & 10 & 1.21272 & 0 & 3.36866 \\
``N=50'' & 50 & 0.91199 & 0 & 12.66652 \\ 
``$\chi_{pn} = 1.2 \chi_{c}$'' & 20 & 0.89833 & 0 & 2.49536 \\
``$\chi_{pn} = 1.6 \chi_{c}$'' & 20 & 1.19777 & 0 & 9.98142 \\ 
``$\chi_{ns} = 0.6 \chi_{c}$'' & 20 & 1.04805 & 0.6 & 5.82250 \\
``$\chi_{ns} = 1.2 \chi_{c}$'' & 20 & 1.04805 & 1.2 & 5.82250 \\
\hline
\end{tabular}
\egroup
\end{table}

% 65 pts
\begin{longtable}{cccc|cccc}
\caption{Coarse resolution sweep of the volume fraction of the film initial condition. 
The initial bath concentration was always set to: $\phi_{p} = 0.01$, $\phi_{n} = 0.98$ and $\phi_{s} = 0.01$.
}\\
\label{tab:coarse}\\
\hline
Run no. & $\phi_{p}$ & $\phi_{n}$ & $\phi_{s}$ & Run no. & $\phi_{p}$ & $\phi_{n}$ & $\phi_{s}$ \\
\hline
0 & 0.010 & 0.010 & 0.980  & 33 & 0.301 & 0.301 & 0.398 \\
1 & 0.010 & 0.107 & 0.883  & 34 & 0.301 & 0.398 & 0.301 \\
2 & 0.010 & 0.204 & 0.786  & 35 & 0.301 & 0.495 & 0.204 \\
3 & 0.010 & 0.301 & 0.689  & 36 & 0.301 & 0.592 & 0.107 \\
4 & 0.010 & 0.398 & 0.592  & 37 & 0.301 & 0.689 & 0.010 \\
5 & 0.010 & 0.495 & 0.495  & 38 & 0.398 & 0.010 & 0.592 \\
6 & 0.010 & 0.592 & 0.398  & 39 & 0.398 & 0.107 & 0.495 \\
7 & 0.010 & 0.689 & 0.301  & 40 & 0.398 & 0.204 & 0.398 \\
8 & 0.010 & 0.786 & 0.204  & 41 & 0.398 & 0.301 & 0.301 \\
9 & 0.010 & 0.883 & 0.107  & 42 & 0.398 & 0.398 & 0.204 \\
10 & 0.010 & 0.980 & 0.010 & 43 & 0.398 & 0.495 & 0.107 \\
11 & 0.107 & 0.010 & 0.883 & 44 & 0.398 & 0.592 & 0.010 \\
12 & 0.107 & 0.107 & 0.786 & 45 & 0.495 & 0.010 & 0.495 \\
13 & 0.107 & 0.204 & 0.689 & 46 & 0.495 & 0.107 & 0.398 \\
14 & 0.107 & 0.301 & 0.592 & 47 & 0.495 & 0.204 & 0.301 \\
15 & 0.107 & 0.398 & 0.495 & 48 & 0.495 & 0.301 & 0.204 \\
16 & 0.107 & 0.495 & 0.398 & 49 & 0.495 & 0.398 & 0.107 \\
17 & 0.107 & 0.592 & 0.301 & 50 & 0.495 & 0.495 & 0.010 \\
18 & 0.107 & 0.689 & 0.204 & 51 & 0.592 & 0.010 & 0.398 \\
19 & 0.107 & 0.786 & 0.107 & 52 & 0.592 & 0.107 & 0.301 \\
20 & 0.107 & 0.883 & 0.010 & 53 & 0.592 & 0.204 & 0.204 \\
21 & 0.204 & 0.010 & 0.786 & 54 & 0.592 & 0.301 & 0.107 \\
22 & 0.204 & 0.107 & 0.689 & 55 & 0.592 & 0.398 & 0.010 \\
23 & 0.204 & 0.204 & 0.592 & 56 & 0.689 & 0.010 & 0.301 \\
24 & 0.204 & 0.301 & 0.495 & 57 & 0.689 & 0.107 & 0.204 \\
25 & 0.204 & 0.398 & 0.398 & 58 & 0.689 & 0.204 & 0.107 \\
26 & 0.204 & 0.495 & 0.301 & 59 & 0.689 & 0.301 & 0.010 \\
27 & 0.204 & 0.592 & 0.204 & 60 & 0.786 & 0.010 & 0.204 \\
28 & 0.204 & 0.689 & 0.107 & 61 & 0.786 & 0.107 & 0.107 \\
29 & 0.204 & 0.786 & 0.010 & 62 & 0.786 & 0.204 & 0.010 \\
30 & 0.301 & 0.010 & 0.689 & 63 & 0.883 & 0.010 & 0.107 \\
31 & 0.301 & 0.107 & 0.592 & 64 & 0.883 & 0.107 & 0.010 \\
32 & 0.301 & 0.204 & 0.495 & 65 & 0.980 & 0.010 & 0.010 \\
\hline
\end{longtable}

\begin{longtable}{cccc|cccc}
\caption{Fine resolution sweep of the initial volume fraction of the film.
The initial bath concentration was always set to: $\phi_{p} = 0.01$, $\phi_{n} = 0.98$ and $\phi_{s} = 0.01$.
}\\
\label{tab:fine}\\
\hline
Run no. & $\phi_{p}$ & $\phi_{n}$ & $\phi_{s}$ & Run no. & $\phi_{p}$ & $\phi_{n}$ & $\phi_{s}$ \\
\hline
0 & 0.010 & 0.010 & 0.980  &  116 & 0.301 & 0.253 & 0.446 \\
1 & 0.010 & 0.059 & 0.931  &  117 & 0.301 & 0.301 & 0.398 \\
2 & 0.010 & 0.107 & 0.883  &  118 & 0.301 & 0.350 & 0.349 \\
3 & 0.010 & 0.156 & 0.835  &  119 & 0.301 & 0.398 & 0.301 \\
4 & 0.010 & 0.204 & 0.786  &  120 & 0.301 & 0.447 & 0.252 \\
5 & 0.010 & 0.253 & 0.738  &  121 & 0.301 & 0.495 & 0.204 \\
6 & 0.010 & 0.301 & 0.689  &  122 & 0.301 & 0.543 & 0.155 \\
7 & 0.010 & 0.350 & 0.640  &  123 & 0.301 & 0.592 & 0.107 \\
8 & 0.010 & 0.398 & 0.592  &  124 & 0.301 & 0.641 & 0.058 \\
9 & 0.010 & 0.447 & 0.543  &  125 & 0.301 & 0.689 & 0.010 \\
10 & 0.010 & 0.495 & 0.495  & 126 & 0.350 & 0.010 & 0.640 \\
11 & 0.010 & 0.543 & 0.447  & 127 & 0.350 & 0.059 & 0.592 \\
12 & 0.010 & 0.592 & 0.398  & 128 & 0.350 & 0.107 & 0.543 \\
13 & 0.010 & 0.641 & 0.349  & 129 & 0.350 & 0.156 & 0.495 \\
14 & 0.010 & 0.689 & 0.301  & 130 & 0.350 & 0.204 & 0.446 \\
15 & 0.010 & 0.738 & 0.252  & 131 & 0.350 & 0.253 & 0.398 \\
16 & 0.010 & 0.786 & 0.204  & 132 & 0.350 & 0.301 & 0.349 \\
17 & 0.010 & 0.835 & 0.155  & 133 & 0.350 & 0.350 & 0.301 \\
18 & 0.010 & 0.883 & 0.107  & 134 & 0.350 & 0.398 & 0.252 \\
19 & 0.010 & 0.931 & 0.058  & 135 & 0.350 & 0.447 & 0.204 \\
20 & 0.010 & 0.980 & 0.010  & 136 & 0.350 & 0.495 & 0.155 \\
21 & 0.059 & 0.010 & 0.931  & 137 & 0.350 & 0.543 & 0.107 \\
22 & 0.059 & 0.059 & 0.883  & 138 & 0.350 & 0.592 & 0.058 \\
23 & 0.059 & 0.107 & 0.835  & 139 & 0.350 & 0.641 & 0.010 \\
24 & 0.059 & 0.156 & 0.786  & 140 & 0.398 & 0.010 & 0.592 \\
25 & 0.059 & 0.204 & 0.738  & 141 & 0.398 & 0.059 & 0.543 \\
26 & 0.059 & 0.253 & 0.689  & 142 & 0.398 & 0.107 & 0.495 \\
27 & 0.059 & 0.301 & 0.640  & 143 & 0.398 & 0.156 & 0.446 \\
28 & 0.059 & 0.350 & 0.592  & 144 & 0.398 & 0.204 & 0.398 \\
29 & 0.059 & 0.398 & 0.543  & 145 & 0.398 & 0.253 & 0.349 \\
30 & 0.059 & 0.447 & 0.495  & 146 & 0.398 & 0.301 & 0.301 \\
31 & 0.059 & 0.495 & 0.447  & 147 & 0.398 & 0.350 & 0.252 \\
32 & 0.059 & 0.543 & 0.398  & 148 & 0.398 & 0.398 & 0.204 \\
33 & 0.059 & 0.592 & 0.349  & 149 & 0.398 & 0.447 & 0.155 \\
34 & 0.059 & 0.641 & 0.301  & 150 & 0.398 & 0.495 & 0.107 \\
35 & 0.059 & 0.689 & 0.252  & 151 & 0.398 & 0.543 & 0.058 \\
36 & 0.059 & 0.738 & 0.204  & 152 & 0.398 & 0.592 & 0.010 \\
37 & 0.059 & 0.786 & 0.155  & 153 & 0.447 & 0.010 & 0.543 \\
38 & 0.059 & 0.835 & 0.107  & 154 & 0.447 & 0.059 & 0.495 \\
39 & 0.059 & 0.883 & 0.058  & 155 & 0.447 & 0.107 & 0.447 \\
40 & 0.059 & 0.931 & 0.010  & 156 & 0.447 & 0.156 & 0.398 \\
41 & 0.107 & 0.010 & 0.883  & 157 & 0.447 & 0.204 & 0.349 \\
42 & 0.107 & 0.059 & 0.835  & 158 & 0.447 & 0.253 & 0.301 \\
43 & 0.107 & 0.107 & 0.786  & 159 & 0.447 & 0.301 & 0.252 \\
44 & 0.107 & 0.156 & 0.738  & 160 & 0.447 & 0.350 & 0.204 \\
45 & 0.107 & 0.204 & 0.689  & 161 & 0.447 & 0.398 & 0.155 \\
46 & 0.107 & 0.253 & 0.641  & 162 & 0.447 & 0.447 & 0.107 \\
47 & 0.107 & 0.301 & 0.592  & 163 & 0.447 & 0.495 & 0.058 \\
48 & 0.107 & 0.350 & 0.543  & 164 & 0.447 & 0.543 & 0.010 \\
49 & 0.107 & 0.398 & 0.495  & 165 & 0.495 & 0.010 & 0.495 \\
50 & 0.107 & 0.447 & 0.447  & 166 & 0.495 & 0.059 & 0.447 \\
51 & 0.107 & 0.495 & 0.398  & 167 & 0.495 & 0.107 & 0.398 \\
52 & 0.107 & 0.543 & 0.350  & 168 & 0.495 & 0.156 & 0.349 \\
53 & 0.107 & 0.592 & 0.301  & 169 & 0.495 & 0.204 & 0.301 \\
54 & 0.107 & 0.641 & 0.252  & 170 & 0.495 & 0.253 & 0.253 \\
55 & 0.107 & 0.689 & 0.204  & 171 & 0.495 & 0.301 & 0.204 \\
56 & 0.107 & 0.738 & 0.155  & 172 & 0.495 & 0.350 & 0.155 \\
57 & 0.107 & 0.786 & 0.107  & 173 & 0.495 & 0.398 & 0.107 \\
58 & 0.107 & 0.835 & 0.058  & 174 & 0.495 & 0.447 & 0.058 \\
59 & 0.107 & 0.883 & 0.010  & 175 & 0.495 & 0.495 & 0.010 \\
60 & 0.156 & 0.010 & 0.835  & 176 & 0.543 & 0.010 & 0.447 \\
61 & 0.156 & 0.059 & 0.786  & 177 & 0.543 & 0.059 & 0.398 \\
62 & 0.156 & 0.107 & 0.738  & 178 & 0.543 & 0.107 & 0.350 \\
63 & 0.156 & 0.156 & 0.689  & 179 & 0.543 & 0.156 & 0.301 \\
64 & 0.156 & 0.204 & 0.641  & 180 & 0.543 & 0.204 & 0.253 \\
65 & 0.156 & 0.253 & 0.592  & 181 & 0.543 & 0.253 & 0.204 \\
66 & 0.156 & 0.301 & 0.543  & 182 & 0.543 & 0.301 & 0.155 \\
67 & 0.156 & 0.350 & 0.495  & 183 & 0.543 & 0.350 & 0.107 \\
68 & 0.156 & 0.398 & 0.447  & 184 & 0.543 & 0.398 & 0.058 \\
69 & 0.156 & 0.447 & 0.398  & 185 & 0.543 & 0.447 & 0.010 \\
70 & 0.156 & 0.495 & 0.350  & 186 & 0.592 & 0.010 & 0.398 \\
71 & 0.156 & 0.543 & 0.301  & 187 & 0.592 & 0.059 & 0.349 \\
72 & 0.156 & 0.592 & 0.252  & 188 & 0.592 & 0.107 & 0.301 \\
73 & 0.156 & 0.641 & 0.204  & 189 & 0.592 & 0.156 & 0.252 \\
74 & 0.156 & 0.689 & 0.155  & 190 & 0.592 & 0.204 & 0.204 \\
75 & 0.156 & 0.738 & 0.107  & 191 & 0.592 & 0.253 & 0.155 \\
76 & 0.156 & 0.786 & 0.058  & 192 & 0.592 & 0.301 & 0.107 \\
77 & 0.156 & 0.835 & 0.010  & 193 & 0.592 & 0.350 & 0.058 \\
78 & 0.204 & 0.010 & 0.786  & 194 & 0.592 & 0.398 & 0.010 \\
79 & 0.204 & 0.059 & 0.738  & 195 & 0.641 & 0.010 & 0.349 \\
80 & 0.204 & 0.107 & 0.689  & 196 & 0.641 & 0.059 & 0.301 \\
81 & 0.204 & 0.156 & 0.641  & 197 & 0.641 & 0.107 & 0.252 \\
82 & 0.204 & 0.204 & 0.592  & 198 & 0.641 & 0.156 & 0.204 \\
83 & 0.204 & 0.253 & 0.544  & 199 & 0.641 & 0.204 & 0.155 \\
84 & 0.204 & 0.301 & 0.495  & 200 & 0.641 & 0.253 & 0.107 \\
85 & 0.204 & 0.350 & 0.447  & 201 & 0.641 & 0.301 & 0.058 \\
86 & 0.204 & 0.398 & 0.398  & 202 & 0.641 & 0.350 & 0.010 \\
87 & 0.204 & 0.447 & 0.350  & 203 & 0.689 & 0.010 & 0.301 \\
88 & 0.204 & 0.495 & 0.301  & 204 & 0.689 & 0.059 & 0.252 \\
89 & 0.204 & 0.543 & 0.253  & 205 & 0.689 & 0.107 & 0.204 \\
90 & 0.204 & 0.592 & 0.204  & 206 & 0.689 & 0.156 & 0.155 \\
91 & 0.204 & 0.641 & 0.155  & 207 & 0.689 & 0.204 & 0.107 \\
92 & 0.204 & 0.689 & 0.107  & 208 & 0.689 & 0.253 & 0.058 \\
93 & 0.204 & 0.738 & 0.058  & 209 & 0.689 & 0.301 & 0.010 \\
94 & 0.204 & 0.786 & 0.010  & 210 & 0.738 & 0.010 & 0.252 \\
95 & 0.253 & 0.010 & 0.738  & 211 & 0.738 & 0.059 & 0.204 \\
96 & 0.253 & 0.059 & 0.689  & 212 & 0.738 & 0.107 & 0.155 \\
97 & 0.253 & 0.107 & 0.641  & 213 & 0.738 & 0.156 & 0.107 \\
98 & 0.253 & 0.156 & 0.592  & 214 & 0.738 & 0.204 & 0.058 \\
99 & 0.253 & 0.204 & 0.544  & 215 & 0.738 & 0.253 & 0.010 \\
100 & 0.253 & 0.253 & 0.495 & 216 & 0.786 & 0.010 & 0.204 \\
101 & 0.253 & 0.301 & 0.447 & 217 & 0.786 & 0.059 & 0.155 \\
102 & 0.253 & 0.350 & 0.398 & 218 & 0.786 & 0.107 & 0.107 \\
103 & 0.253 & 0.398 & 0.350 & 219 & 0.786 & 0.156 & 0.058 \\
104 & 0.253 & 0.447 & 0.301 & 220 & 0.786 & 0.204 & 0.010 \\
105 & 0.253 & 0.495 & 0.253 & 221 & 0.835 & 0.010 & 0.155 \\
106 & 0.253 & 0.543 & 0.204 & 222 & 0.835 & 0.059 & 0.107 \\
107 & 0.253 & 0.592 & 0.155 & 223 & 0.835 & 0.107 & 0.058 \\
108 & 0.253 & 0.641 & 0.107 & 224 & 0.835 & 0.156 & 0.010 \\
109 & 0.253 & 0.689 & 0.058 & 225 & 0.883 & 0.010 & 0.107 \\
110 & 0.253 & 0.738 & 0.010 & 226 & 0.883 & 0.059 & 0.058 \\
111 & 0.301 & 0.010 & 0.689 & 227 & 0.883 & 0.107 & 0.010 \\
112 & 0.301 & 0.059 & 0.640 & 228 & 0.931 & 0.010 & 0.059 \\
113 & 0.301 & 0.107 & 0.592 & 229 & 0.931 & 0.059 & 0.010 \\
114 & 0.301 & 0.156 & 0.543 & 230 & 0.980 & 0.010 & 0.010 \\
115 & 0.301 & 0.204 & 0.495 &       &       &       &       \\
\hline
\end{longtable}

\begin{figure}
\includegraphics[width=6.5in]{figS1-ic_fractions}
\caption{(a) A coarse sweep and (b) a fine sweep of the composition of the film initial condition.}
\label{fig-ic_fractions}
\end{figure}

%2D periodic runs:
%(from 231 data set)
%100, 101, 102, 103, 104
%117, 118, 119
%32, 33
%47, 48, 49, 50, 51, 52
%65, 66, 67, 68, 69, 70
%84, 85, 86, 87
%
%3D periodic runs:
%100, 101, 102, 103, 104
%117, 118, 119
%84, 85, 86, 87
%
%2D late time runs:
%11, 12, 13, 14, 15, 16, 17, 18, 19
%21, 22, 23, 24, 25, 26, 27, 28, 30
%31, 32, 33, 34, 35, 26
%66, 67, 68, 69, 70, 71, 72, 73, 74, 75, 76, 77, 78, 79, 
%80, 81, 82, 83, 84, 85, 86, 87, 88, 89, 90, 91, 92, 93, 94, 95, 96
%
%3D late time runs:
%?
% }}}

\section{Parameter Studies of the Early-Time Regimes}
% {{{
One can gain additional insight into the kinetic regimes by examining the effects of model parameters.
Accordingly, Figure~\ref{fig-param_study} categorizes the early-time kinetic regimes of several different data sets by their initial condition by systematically varying $N_{p}$, $\chi_{pn}$ and $\chi_{ns}$.
Note that these parameters affect both the phase diagram as well as kinetics parameters, such as the mutual diffusion coefficient~\cite{Tree2017}.
We will briefly examine the effect of each parameter in turn.

\begin{figure*}[tbp]
  \includegraphics[width=6.5in]{figS2-param_study}
  \caption{Characterization of the early-time kinetic regime by initial condition for the seven different sets of parameters labeled in the figure.
Note that: panel (d) is the base case, $\chi_{ns}$ varies in the upper left panels (a, c, d), $N_{p}$ varies in the upper right panels (b, d, e) and $\chi_{pn}$ varies in the bottom panels (d, f, g).
Conditions which lead to regime I (no phase separation) appear as red dots, regime II (phase separation) appear as green dots and regime III (immediate spinodal decomposition) appear as blue dots.}
  \label{fig-param_study}
\end{figure*}

Figure~\ref{fig-param_study}(a), \ref{fig-param_study}(c) and \ref{fig-param_study}(d) show results from three data sets where the solvent/nonsolvent interaction parameter, $\chi_{ns}$, varies from 0 to $1.2$, at constant $N_{p}$ and $\chi_{np}$.
The phase diagram in these cases shifts only slightly, and one observes little change in the location of regimes I and II with changing $\chi_{ns}$.
These observations agree with our earlier conclusion that the local concentration of the interface plays a dominant role in distinguishing between regimes I and II/III.
By contrast, regime III is quite sensitive to the change in the interaction parameter, and shrinks considerably as $\chi_{ns}$ increases.
We hypothesize that a change in the rate of solvent/nonsolvent exchange is the primary cause. 
Increasing $\chi_{ns}$ decreases the nonsolvent diffusion relative to polymer diffusion and leads to a smaller ``tail'' on the film side of the composition path, which in turn gives a smaller region for regime III.

Figure~\ref{fig-param_study}(b), \ref{fig-param_study}(d) and \ref{fig-param_study}(e) show data sets where the polymer degree of polymerization, $N_{p}$, varies from 10 to 50, $\chi_{ns} = 0$ and $\chi_{np}$ is held at a constant ratio to its binary critical value.
In this case, the phase diagram narrows, and shifts towards smaller polymer concentrations as $N_{p}$ increases.
The change in the boundary between regime I and regimes II/III is again modest, with a small increase in the region of the composition space taken by II and III as $N_{p}$ increases.
The most notable difference however, is the shift of regime III to the right, which mirrors the movement of the spinodal region as the phase diagram shifts.
This shift is again consistent with our observation that the tail of the composition path due to the polymer film must cross into the spinodal for regime III to occur.

Figure~\ref{fig-param_study}(d), \ref{fig-param_study}(f) and \ref{fig-param_study}(g) show data where $N_{p}$ and $\chi_{ns}$ are held constant, while $\chi_{np}$ varies from $1.2 \chi_{c}$ to $1.6 \chi_{c}$ where 
\begin{equation}
\chi_{c} = \frac{1}{2}\left ( \frac{1}{ N_{n}^{1/2} } + \frac{1}{ N_{p}^{1/2} }\right )^{2}
\end{equation}
is the critical Flory Huggins parameter for a binary mixture of polymer and nonsolvent.
As $\chi_{np}$ increases, the quench deepens and the two-phase region expands to cover a larger amount of the composition space.
Concomitantly, the size of regimes II and III also increase, and by $\chi_{np} = 1.6 \chi_{c}$, there is only a small region of regime I remaining.
% }}}

\section{Time Invariance When Films Precipitate}
% {{{
In the main text, we claim that at late times, the system dynamics can be characterized by two dimensionless numbers, $\xi$ and $t D_{0}/l_{f}^{2}$.
We use Figure~\ref{main:fig-time_invariance} as evidence for this claim, since both real space and composition space curves collapse when scaled appropriately.
Figure~\ref{fig-time_invar_spinodal} provides further evidence, showing a similar calculation (parameters $N_{p} = 20$, $\chi_{pn} = 1.048$, $\kappa = 2$) chosen at a different initial condition: $\{\phi_{p}^{\txtf}, \phi_{n}^{\txtf}\} = \{0.204, 0.301\}$ and $\{\phi_{p}^{\txtb}, \phi_{n}^{\txtb}\} = \{0.01, 0.98\}$.

Unlike the figure in the main text, at this initial condition, the system undergoes spinodal decomposition at long times (i.e.\ late time regime III).
Even so, one still observes from the figure that when the length and time are appropriately scaled, the diffusive transport remains identical.
Notably, in the inset in Figure~\ref{fig-time_invar_spinodal}(b) there is not an exact correspondance in the domain location following surface directed spinodal decompostion.
This is to be expected since the phase separation kinetics add additional time and length scales, and are also dependent on the stochastic nature of the initial condition.

\begin{figure*}[tbp]
  \includegraphics[width=6.5in]{figS3-time_invar_spinodal}
  \caption{Comparison of the long time diffusion dynamics between two differnet film thicknesses in (a) composition space and (b) real space. 
The solid curves show a film thickess of $l_{\txtf} = 102.4$ at time points (i) 10, (ii) 500, (iii) $2\times10^{3}$, and (iv) $5\times10^{3}$ (in units of the Rouse time).
The closed circles show a film thickness of $l_{\txtf} = 51.2$ at time points (i) 2.5, (ii) 125, (iii) 500, and (iv) $1.25\times10^{3}$
The inset depicts a zoomed-in portion of the real space volume fraction profile curve (iii).}
  \label{fig-time_invar_spinodal}
\end{figure*}
% }}}

% Results we had some incomplete data for that am omitting:
% - Parameter study of the late-time regimes
% - Data/Plot of initial domain size after phase separation for early-time
% - Data/Plot of initial domain size after phase separation for late-time
% - Data/Plot of domain size versus time (coarsening)

\clearpage
\bibliography{refs}

\end{document}

